%---------------------------------------------------------------------------- 
\chapter{Hívásfelépítés Kubernetes nélkül}
%---------------------------------------------------------------------------- 

Ahhoz, hogy a valóshoz hasonló infrastruktúrát tudjak teremteni VirtualBox-t
használtam, amivel virtuális gépeket lehet létrehozni, amik között különböző 
hálózati kapcsolatokat lehet kialakítani. Esetemben a Bridged adapter volt a 
legkézenfekvőbb, mert az így létrehozott virtuális gépek látszanak minden 
eszköznek a lokális hálózaton. Így képes voltam hívást kezdeményezni a 
hálózatban található két eszköz között. 

A híváshoz szükséges telepíteni valamilyen softphone alkalmazást mind a hívó
feleknek, amikbe be kell jelentkezni egy már meglévő SIP fiókkal. Erre a célra
tökéletesen megfelelt a Linphone, ami ingyenes és nagyon egyszerű konfigurálni. 

A rendszer kiépítéséhez elsőnek szükségem volt egy SIP szerverre, amire a 
Kamailio-t választottam, mivel az rtpengine-t alapvetően ehhez ajánlják. 
Majd létrehoztam egy második szervert, amin az rtpengine futott. 

\section{rtpengine}

Az rtpengine egy RTP (Real-time Transport Protocol) és UDP proxy, amivel könnyen lehet
a forgalmat irányítani illetve transzkódolni. A Kamailio SIP proxy-hoz lett tervezve, de
könnyen lehet más rendszerekkel is használni.

Főbb funkció közé tartoznak az alábbiak, amiken kívül még nagyon sok mással is rendelkezi: 

\begin{itemize}
	\item Tud IPv4 és IPv6 címeket kezelni és közöttük média forgalmat továbbítani. 
	\item Állítható port tartomány. 
	\item Több interfész használata. 
	\item Kernel szintű csomagtovábbítás a kisebb késleltetés és processzor használat miatt.
	\item HTTP, HTTPS és WebSocket interfész támogatottság.
	\item Médiafolyamok felvétele. 
	\item Híváshoz szükséges statisztikák számítása.
	\item Transzkódolás és újracsomagolás.
\end{itemize}

\subsection{Kernel és felhasználói tér}

Az rtpengine alapvetően a felhasználói térben működik így ilyenkor mindent ott csinál, ami
magasabb késeltetést és processzor használatot igényel. Az oka ennek, hogy mikor egy csomag
érkezik a hálózati interfészre az áthalad a kernelen a fájlleíróig. Ha odajutott egy 
csomag, akkor a hozzá kapcsolódó folyamatot értesíti, ami ebben az esetben az rtpengine, hogy
van mit leolvasni onnét. Ennek során a csomag átkerül a kernelből a felhasználóit térbe és 
ez egy nagyon drága művelet. Ha feldolgozásra került a csomag, akkor ugyanezen az útvonalon
fog kimenni. Ez azért nagy probléma, mert az RTP protokoll UDP-t használ és így rengeteg 
kis csomag érkezik folyamatosan, ami nagy terhelést jelent a szerver számára. 

Ezzel szemben, ha használjuk a megfelelő modult, akkor a beérkező csomagok nem kerülnek át a 
felhasználói térbe így kevesebb erőforrás kerül felhasználásra. Ezt úgy éri el az rtpengine,
hogy egy hívás kiépítése során \textbf{iptables} szabályokat hoz létre, amikben definiálja, hogy
adott portra érkező forgalom hova legyen továbbítva. 

Az \textbf{iptables} egy olyan eszköz, amivel a linux kernelben lehet létrehozni 
táblákat szabályokkal, amik IP szinten képesek szűrni a forgalmat. 

%A választás azért erre a proxy-ra esett, mert folyamatos támogatottsága van, sok helyen 
%használják illetve sok olyan funkcióval rendelkezik, ami a projekt szempontjából fontos.

%Ilyen és legszükségesebb funkciója a Redis támogatottsága, amivel egyes hívásokhoz 
%lehet minden szükséges adatot elmenteni és ezen adatok alapján más rtpengine példányok 
%tudják frissíteni a hívásaikat. Így, ha egy az egyik szerver megáll és nem tudja tovább
%kezelni a hívásokat, akkor egy terhelés elosztó áttudja irányítani az összes hívást 
%más szerverekhez, amik ezáltal rendelkeznek a híváshoz szükséges információkkal. 
%
%A Redis támogatottság már létezett, viszont csak akkor, ha az rtpengine konfigurálásánál
%pontosan ismerjük az összes többi rtpengine címét. Alapesetben ezek az információk 
%rendelkezésre állnak, viszont Kubernetes környezetben ezek mind változékonyak. Sosem 
%tudjuk pontosan megjósolni, hogy egy kapszula, ami egy rtpengine szervert futtat milyen
%címmel fog létrejönni. 
%
%Ezt a problémát orvosolta részben Oded Arbell, aki egy Pull Request keretein belül 
%benyújtott egy nulla tudású Redis skálázhatóság funkciót. Mint a neve is utal rá 
%ezzel a megoldással úgy lehet az infrastruktúrába új rtpengine szervereket felvenni, 
%hogy nem kell ismerni a többi résztvevő szerver címét. De ez még mindig csak részben 
%oldotta meg a problémát ugyanis ha van két rtpengine szerverünk, akkor az egyikhez 
%beérkező hívásnál csak azon a szerveren fognak kinyílni a megfelelő portok, amire a 
%hívás beérkezett. Így, ha ez a szerver leáll, akkor a másik nem tudja kezelni a forgalmat,
%mert nincs nyitva egyik szükséges portja sem. 
%
%Megoldásképpen egy kicsit bele kellett nyúlnom ebbe a kódba és beállítani, hogy frissítésnél
%nyisson ki minden olyan portot, ami a hívás információi között szerepel. Így elértem 
%egy olyan fajta redundanciát, amivel a fentebb leírt váltás jól működik. \\
%
%Amit még érdemes megemlíteni, hogy az rtpengine képes kettő különböző módban működni.
%Tud simán a felhasználói névtérben futni, amikor a transzkódolást és az adatok 
%továbbítását is ott végzi el és tudja a kernel névteret is használni, amikor a 
%továbbítást a kernel névtérben végzi. Így processzor erőforrást lehet spórolni, mert
%nem kell a szervernek azt a drága műveletet minden csomagnál elvégeznie, hogy 
%a kernel névtérből felhasználóiba teszi át a csomagokat. 

\subsection{Transzkódolás}

Transzkódolás alatt azt a folyamatot értjük, amikor egy adott formátumú videó- vagy 
hanganyag egy másik formátumba való átalakítása történik. Ezt jelenleg az rtpengine 
csak hanggal tudja megvalósítani. 

Alapvetően elérhető ez a funkció, de ki is kapcsolható teljesen. Ugyanakkor az, hogy 
elérhető az rtpengine nem avatkozik bele a hívott felek közötti kodek megegyezésbe, 
így ha nincs közös kodek-je a két félnek, akkor a hívás sikertelen lesz.

A transzkódlást az ng vezérlőprotokoll \textit{transcoding} vagy \textit{ptime} beállításával
lehet az rtpengine tudtára adni. Így, ha a két félnek nincs közös kodekje, de az rtpengine
mind a kettő számára képes egy olyat nyújtani, amit elfogad, akkor a hívás sikeresen kiépül.

Ha aktív transzkódolás van, akkor az SDP-ből kikerül minden olyan kodek, amit nem támogatnak
a kliensek. Emellett ezt nem lehet kernel szinten továbbítani, ezért mindenképpen a 
felhasználói térben lesznek ezek a csomagok feldolgozva. \\

Ezek a formátumok vannak támogatva: G.711 (a-Law és µ-Law), G.722, G.723.1, G.729, Speex, 
GSM, iLBC, Opus, AMR (keskeny és széles sávú)

\subsection{ng vezérlőprotokoll}

Ahhoz, hogy bizonyos funkciók engedélyezve legyenek az rtpengine-ben kifejlesztettek 
egy vezérlési protokollt, amibe a SIP proxy beágyazva áttudja adni az SDP törzset 
azt rtpengine-nek, majd az ezt a törzset módosítva küldi vissza. 

Ez a vezérlőprotokoll bencode formátumban küldi ki el az üzeneteket. Ez nagyon hasonló
felépítéssel rendelkezik, mint a JSON, viszont sokkal egyszerűbb és gyorsabb a kódolása 
és dekódolása. Ami hátránya a JSON-l szemben, hogy kevésbé olvasható. Fontos 
megjegyezni, hogy ilyen üzenetek képes fogadni az rtpengine ezeken a protokollokon: 
UDP, TCP, HTTP, HTTPS, WebSocket, WebSocket Secure. 

Minden ilyen üzenet két részből áll. Egy sütiből, ami ebben az esetben azonosítóként
funkciónál és egy bencode szótárból, ami legalább egy parancsot tartalmaz az 
alábbiak közül. A felsorolás nem teljes, csak a téma szempontjából fontosokat 
emeltem ki.

\begin{itemize}
	\item \textbf{ping}: Ellenőrizhető, hogy az rtpengine elérhető-e. Egyetlen helyes 
	válasz van rá, az pedig a \textbf{pong}.
	\item \textbf{offer}: A hívó fél adatit rögzíti.
	\item \textbf{answer}: A hívott fél adatit rögzíti. 
	\item \textbf{delete}: Egy adott azonosítóval rendelkező hívást lehet törölni.
	\item \textbf{query}: Egy hívás részleteit lehet lekérdezni. 
	\item \textbf{statistics}: Egy adott azonosítóval rendelkező hívásról lehet statisztikát
	lekérdezni. 
\end{itemize} 

\subsection{Beállítás}

Elsőnek az rtpengine-t kell beállítani, amihez a következő konfigurációt 
használom: 

\begin{lstlisting}
	[rtpengine]
	interface=192.168.1.106
	foreground=true
	log-stderr=true
	listen-ng=192.168.1.106:22222
	port-min=23000
	port-max=32768
	log-level=7
\end{lstlisting}

Ezzel a következő dolgokat állítom be: 

\begin{itemize}
	\item \textbf{interface}: Megadja az RTP helyi hálózati interfészét. Legalább
	egyet meg kell adni, de több is megadható.
	\item \textbf{listen-ng}: A vezérlő üzeneteket ezen a címen várja. 
	\item \text{port-min és port-max}: Az UDP média forgalom számára lefoglalható
	portok tartománya. 
	\item \textbf{Többi}: Naplózás szempontjából érdekesek csak. 
\end{itemize}

\section{Kamailio}

A Kamailio egy SIP Signaling Server nyílt forráskódú megvalósítása. Amit nagy rendszerek
skálázhatóságára terveztek, de használható cégek és egyének számára is.

A használatához általában szükség van valamilyen Kamailio által támogatott adatbázishoz,
de elméletileg lehet adatbázis nélkül is használni, de akkor elvesztünk bizonyos funkciókat. 
Ezért az ajánlott adatbázist használtam hozzá, ami egy lokálisan futó mysql szerver. Ebbe 
az adatbázisba olyan táblákat tárol, mint regisztrált felhasználók és aktív hívások, de 
ezen felül még sok mást is. 

Ezenkívül az általam használt beállításokat a \textit{kamailio.cfg} fájlban lehet megadni,
ami három részből áll. 

Az első a \textbf{globális paraméterek}, ahol leginkább környezeti változókat lehet 
definiálni, amikkel lehet kontrollálni, hogy a konfiguráció mely része fusson le illetve
értékeket is lehet hozzájuk rendelni. 

Aztán jönnek a \textbf{modul beállítások}, ahol az alapértelmezett modulokhoz tudunk plusz
modulokat betölteni és beállítani. Ez azért egy fontos rész, mert a Kamailio alapvetően 
nem rtpengine-re van beállítva, hanem annak az előző verziójához, ami az rtpproxy. 
Az rtpproxy-t nem kell telepíteni, mert ez kamailio csomagjai közt szerepel. 

Végül van az irányítási blokk, ahol különböző SIP üzenetekre lehet megmondani, hogy mi
történjen velük, ha beérkeznek a hálózati interfészre. Itt betudjuk állítani azt, hogy
az \textbf{INVITE} üzenetek esetén az ng vezérlőprotokollon keresztül állítsa be a hívást
az rtpengine szerveren.

\section{Kommunikáció}

