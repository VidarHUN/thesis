%----------------------------------------------------------------------------   
\chapter{Kliens}
%----------------------------------------------------------------------------

Ahhoz, hogy az előzőleg tárgyalt rendszert érdemben lehessen tesztelni írnom
kellett egy saját alkalmazást, ami képes volt adott számú médiaforgalmat 
generálni, de tud kommunikálni az rtpengine-l is. Szóval egyben a SIP
klienst és szervert is megvalósítja.

A kliens felépítése nagyon hasonlóan néz ki, mint a kontrolleré, ami azt 
eredményezi, hogy képesnek kell lennie  TCP, UDP és WebSocket protokollokon
keresztül kommunikálni az rtpengine-l. Viszont a klienshez nem érkeznek be 
vezérlőparancsok, hanem a kliens gyártja ezeket. Így a kliensnek tudnia kell
az összes rtpengine által támogatott vezérlőparancsot megalkotnia bizonyos 
paraméterekkel. Ezen felül tudnia kell médiaforgalmat is generálni, amihez
jelenleg két fajta megoldás létezik. Használható az FFmpeg és az rtpsend 
is arra, hogy ezt a forgalmat legenerálja, viszont mind a kettőnek megvan
az előnye és a hátránya is. 

Ebben a fejezetben kifejtésre kerül a kliensben használt külső alkalmazások
működése és célja, a kliens használata és annak felépítése. 

\section{FFmpeg \& rtpsend}

Elsőnek az FFmpeg-t szeretném bemutatni, mert ezt az eszközt többen ismerhetik,
mint az rtpsend-t. 

Az FFmpeg \cite{ffmpeg} egy nyílt forráskódú szoftver, amely különböző könyvtárak felhasználásával
képes videó, hang és egyéb multimédiás adatfolyamok kezelésére. Így olyan feladatokat
lehet vele ellátni többek között, mint kódolás, dekódolás, transzformálás és 
adatfolyam sugárázása. Így egy olyan feladat ellátása, hogy adott hangfájl sugárzása a 
hálózaton keresztül könnyen megvalósítható. Viszont rendelkezik különböző hátrányokkal
amik nem elhanyagolhatóak a teszt szempontjából.

Az első ilyen probléma vele, hogy az erőforrás igénye nagyon magas tud lenni 
annak függvényében, hogy mennyire komplex feladatot lát el. Ami alapesetben nem lenne
probléma, viszont ebben a környezetben sok egymás mellett futó FFmpeg végez majd
komplex feladatot, amihez így sok erőforrás szükséges.

A második probléma, hogy a sugárzott RTP folyamot nem lehet eléggé testre szabni, 
amennyire szükséges lenne. Így például a csomagokban lévő adat nem mindig lesz 
ugyanakkora méretű. Ami a mérési eredményeket könnyen torzítani fogja, hiszen 
minden médiafolyam rendelkezik egy bizonyos kódolással, amihez megvan határozva, hogy
milyen méretű csomagok szükségesek.

Végül a legnagyobb probléma, ami előjön az FFmpeg és rtpsend esetében is, hogy nem
tudnak RTCP fogadó jelentést gyártani, ami miatt a hívás minősége nem monitorozható 
megfelelően. Ez a probléma úgy lett kiküszöbölve, hogy rtpsend vagy FFmpeg gyárt
egy adott mennyiségű háttérforgalmat és két Linphone kliens segítségével elindításra 
kerül mellettük egy rendes hívás. Így mérhetővé válik, hogy valamekkora háttérforgalom
mellett milyen minőséget tud nyújtani a rendszer. \\

Az alábbi parancs szükséges ahhoz, hogy egy wav fájlt sikeresen lehessen sugározni
RTP segítségével az rtpengine számára, így a híváshoz szükséges adatfolyamot lehet
szimulálni. 

\begin{lstlisting}[caption=FFmpeg RTP folyam indtása, label=lst:FFmpeg]
ffmpeg -re -i audio.wav -ar 8000 -ac 1 -acodec pcm_mulaw \\
-f rtp 'rtp://127.0.0.1:23000?localrtpport=2000'
\end{lstlisting}

Ez a sor az \textit{audio.wav} fájlt fogja a lokális hálózaton lévő rtpengine 23000-s
portjára küldeni a 2000-s portról PCMU kódolással. A következő felsorolás 
tartalmazza részletesen, hogy a paraméterei mit csinálnak.

\begin{itemize}
	\item \textbf{-re -i}: Natív képkockasebességgel való beolvasást tesz lehetővé, 
	így lehet szimulálni egy élő forrást, például egy mikrofont. Emellett lelassítja
	a fájl olvasásának sebességet annyira, hogy hasonlítson az a valós idejű
	adatbeolvasáshoz, ugyanis alapértelmezetten olyan gyorsan olvassa az FFmpeg a
	fájlokat amilyen gyorsan csak tudja és így gyorsabban lesz beolvasva a fájl,
	mint a fájl időtartama. 
	Így lehetővé teszi azt, hogy egy három perces hangfájl ne pár másodperc alatt 
	legyen leküldve a hálózaton, hanem három perc alatt. 
	\item \textbf{-ar}: Beállítja a mintavételezési periódust, ami itt 8000 Hz, ami 
	azt jelenti, hogy másodpercenként 8000 mintát vesz a forrásból. 
	\item \textbf{ac}: Beállítja a használt hangcsatornák számát.
	\item \textbf{-acodec}: Megadja, hogy milyen hangkódolást használjon. Ebben az 
	esetben PCM u-law alapú kódolás történik.
	\item \textbf{-f}: A kimenti fájl formátumát lehet megadni, de ebben az esetben
	nem egy fájl, hanem egy cím van megadva így a hálózaton RTP csomagokban fogja 
	küldeni a kódolt hanganyagot. Ha jobban megnézzük, akkor ez a cím két részből áll,
	ahol az első határozza meg azt, hogy hova küldje a csomagokat. Ez a 
	\textit{127.0.0.1:23000} az rtpengine címe és a híváshoz meghatározott egyik RTP
	port. Míg a második rész a \textit{localrtpport}, amivel meghatározható, hogy
	mely portról küldje ezeket a csomagokat.
\end{itemize}

\subsection{rtpsend}

Az rtpsend \cite{rtpsend} teljesen eltér az FFmpeg-től, hiszen itt nincs semmilyen hang vagy videó
feldolgozás, átalakítás vagy bármi más. Itt egyszerűen csak egy RTP folyam visszajátszása
történik, amit az \textit{rtpdump} eszközzel lehet felvenni. Ez a két szoftver egy 
eszköztárba tartozik, aminek a neve \textit{rtptools}, amit Henning Schulzrinne
alkotott meg, aki a Columbia Egyetem oktatója.

Szóval elsőnek fel kell venni egy híváshoz tartozó RTP folyamot az rtpdump paranccsal,
ami úgy működik, hogy meg kell adni egy címet, amin az RTP csomagokat elkapja és 
ezeket alakítja át egy olyan formátumba, amit később az \textit{rtpsend}-l újra
lehet alkotni. Abból a szempontból előnyös, hogy az rtpengine-nek a médiaformátum
átalakítását jól lehet vele tesztelni, mert fellehet venni két különböző kódolású
hívást és azokat lejátszani egy szimulált híváson belül.

A lejátszáshoz ez a parancs lesz használva: 

\begin{lstlisting}[caption=RTP folyam generálásda rtpsend segítségével, label=lst:rtpsend]
rtpsend -l -s 3002 -f dump.rtp 127.0.0.1/23000
\end{lstlisting}

Ezzel azt az utasítást adjuk ki, hogy folyamatos ismétléssel küldje el a 
\textit{dump.rtp} fájlt a \textit{127.0.0.1:3002} címről a \textit{127.0.0.1:23000}
címre, amin az rtpengine várja az egyik oldaltól a médiafolyamot.

\section{Felépítése}

A kliens felépítése nagyon hasonló, mint a kontrolleré és ez látszik abban is, hogy
ugyan azokat a könyvtárakat használja naplózáshoz, konfiguráció beolvasásához és
argumentumok átadására. Viszont a konfigurációs fájlban más paramétereket lehet 
beállítani. A következő felsorolásban fejtem, hogy mely paraméter milyen értékeket 
vehet fel és mi történik ezek használata során. 

\begin{itemize}
	\item \textbf{local\_address}: Meglehet adni, hogy milyen címről küldje
	ki a vezérlőparancsokat.
	\item \textbf{protocol}: Megadja, hogy milyen protokollt használjon a kliens az
	vezérlőparancsok küldésére. Ez lehet udp, tcp, ws azaz WebSocket.
	\item \textbf{rtpe\_address}: Az rtpengine címe. 
	\item \textbf{rtpe\_port}:Az rtpengine portja.
	\item \textbf{ping}: Ha csak azt akarjuk a klienssel ellenőrizni, hogy az rtpengine
	elérhető, akkor lehet küldeni egy szimpla \textit{ping} üzenet. Ha visszajön egy
	\textit{pong} az rtpengine elérhető. Ha ennek a mezőnek az értéke yes, akkor 
	az előzőleg leírt megtörténik és vége a kliens futásának. 
	\item \textbf{number\_of\_calls}: A generálni kívánt hívások számát lehet megadni. 
	\item \textbf{send\_method}: Megadhatjuk, hogy FFmpeg vagy rtpsend segítségével
	generáljon médiaforgalmat a kliens.
	\item \textbf{wav\_location}: Ha FFmpeg lett beállítva a \textit{send\_method} 
	paraméternél, akkor az ehhez szükséges wav hangfájl elérési útvonala adható meg.
	\item \textbf{rtp\_dump\_location}: Ha rtpsend lett beállítva a 
	\textit{send\_method} paraméternél, akkor az ehhez szükséges rtp fájl elérési 
	útvonala adható meg.
\end{itemize}

Indításánál ugyanúgy a konfigutációs fájlt és a naplózás szintjét lehet beállítani,
mint a kontrollernél. \\

Ami még hasonlóan működik, mint a kontroller esetében az a foglatok kezelése.
Szóval az UDP és TCP protokoll kezelése az teljesen ugyan az, de a WebSocket estében
nincs szerver implementálva csak a kliens. 

\subsection{Médiaforgalom generálása}

Ahogy az előzőekben ki lett fejtve, a médiaforgalom generálására két eszközt 
használ a kontroller az rtpsend-t és az FFmpeg-t. Viszont ezekből tetszőleges
számút kell tudni elindítani attól függően, hogy hány párhuzamos hívást kell 
a kliensnek indítania.

Első próbálkozásnál a szálkezeléssel próbálkoztam, de hamar rá kellett jönnöm,
hogy az csak abban az esetben hasznos, ha python kódot kell konkurensen 
használni. Ezért átváltottam a \textit{subprocess} \cite{subprocess} modulra, amivel külső alkalmazásokat
lehet párhuzamosan használni.

Ahhoz, hogy ezzel a modullal létre lehessen hozni egy folyamatot, ahhoz a \textit{Popen}
konstruktort kell használni, amiben az elindítani kívánt folyamat argumentumait lehet
megadni egy listaként. Mivel egy így létrehozott objektum csak egyetlen folyamatot 
képes elindítani, ezért több ilyen objektumot kell létrehozni más paraméterekkel. Ezért
az FFmpeg és az rtpsend esetében is meg kell találni azokat a részeket, amik minden 
hívásnál változnak és ezek szerint létrehozni ezeket az objektumokat. Az FFmpeg esetében
elég volt a rtpengine címét és a híváshoz lefoglalt portját megadni. Így ezeket a címeket
egy listába rendezve majd a listát iterálva ellehet indítani hívásonként kettő
FFmpeg folyamatot. Az rtpsend esetében is ugyanez a felállás, viszont ott a forrás és 
a cél port két különböző paraméterben szerepel. Ehhez már nem egy listát, hanem egy 
szótárat használtam, ahol az rtpengine címe volt a kulcs és a forrás port volt a hozzárendelt
érték.

\subsection{Hívások generálása}

A hívás generálás során biztosítani kell azt, hogy legyen megfelelő mennyiségű port
nyitva médiaforgalomhoz és mielőtt a médiaforgalom elindul meg kell teremteni a hívást
magán az rtpengine szerveren. Amihez szükség van egy \textit{offer} és \textit{answer}
üzenetpárosra.

Viszont mielőtt még ezek az üzenetek megalkotásra és elküldésre kerülnek azelőtt
a hozzájuk tartozó SDP üzeneteket kell megalkotni. Ezt egy sablon alapján készíti el 
a kontroller, amiben módosítása során csak a munkamenet azonosító helyére kerül
mindig egy véletlenszerű szám és a kapcsolódást leíró paraméter kapja meg a lokális 
címet azaz a hívó félét. Ha ez megtörtént azután már ellehet küldeni elsőnek az offer-t
majd az answer-t. Amik ugyanazzal a hívásazonosítóval és \textit{from-tag}-l rendelkeznek,
viszont az answer esetében meg kell adni egy \textit{to-tag}-t is.

Az így megalkotott hívásról információk kerülnek tárolásra egy listában, ami szükséges
a hívások törléshez. Mert ha vége az adott folyamatoknak, attól még az rtpengine szerveren
él a hívás, csak nem érkezik be hozzá semmilyen adat. Amit lehetne, hagyni, hogy majd 
magától megszűnik, hiszen egy adott idejű inaktivitás után törli. De, ha rövid időn 
belül új tesztet kell elindítani, akkor összeakadhatnak a portok és az azonosítók. Így a 
kontroller olyan esetben, amikor a hívás megszakad küld egy \textit{delete} vezérlőüzenetet
az rtpengine felé hívásonként a hozzájuk tartozó azonosítókkal.

