%----------------------------------------------------------------------------   
\chapter{Kontroller}
%----------------------------------------------------------------------------

A kontrollernek két fő feladata van egy hívás felépítése során. Az egyik, hogy
a vezérlőprotokoll üzeneteit feldolgozza illetve továbbítsa az rtpengine
felé. A másik pedig a híváshoz szükséges l7mp beállításokat hivatott 
megvalósítani.

Ehhez készítettem egy python szkriptet, ami ezeket a feladatokat képes ellátni
különböző protokollokat támogatva. Azért a python lett választva a megvalósításhoz,
mert ezt a nyelvet viszonylag jól ismerem illetve nagyon jó Kubernetes 
támogatottsága van. Így könnyen és gyorsan lehetett haladni az írásával. \\

Fontos megjegyezni, hogy a kontroller nem csak l7mp környezetben való használtra lett
tervezve, hanem Envoy-l is. Így vannak olyan megvalósítások, amik csak Envoy
vagy l7mp környezetben nyernek értelmet. 

\section{Használata}

Mielőtt részleteiben kifejtésre kerülne a kontroller működése, szeretném bemutatni a
használatát.

A kontroller működéséhez biztosítani kell egy konfigurációs fájlt, amiben definiálva
van minden olyan paraméter, ami szerint szeretnénk, ha működne. Ezek a paraméterek
az alábbi felsorolásban olvashatóak.

\begin{itemize}
	\item \textbf{protocol}: Használt protokoll a vezérlőparancsok küldéséhez. Ez lehet
	udp, tcp és ws azaz WebSocket is.
	\item \textbf{rtpe\_address}: Az rtpengine szolgáltatás neve vagy IP címe. 
	\item \textbf{rtpe\_port}: A port amin az rtpengine várja a beérkező parancsokat. 
	\item \textbf{envoy\_address}: Ha Envoy környezetben van használva akkor a menedzsment
	szolgáltatás neve vagy IP címe. 
	\item \textbf{envoy\_port}: Envoy menedzsment szolgáltatás portja. 
	\item \textbf{local\_address}: Lehet állítani, hogy az üzenetek küldése során milyen
	lokális címről küldje. Ez l7mp környezetben a '127.0.0.1' míg Envoy esetében '0.0.0.0'.
	\item \textbf{local\_port}: Meghatározható, hogy a kimenő parancsok mely portról
	induljanak el. 
	\item \textbf{sidecar\_type}: Használt proxy típusa állítható be, ami lehet l7mp és envoy is.
	\item \textbf{without\_jsonsocket}: A fejlesztés során több fajta architektúrát kipróbáltam
	és ennek paraméternek a beállításával lehet váltani a létrehozandó l7mp erőforrások között. 
	\item \textbf{ingress\_address}: A Kubernetes dolgozó csomópontjának az IP címe adható meg.
	Azért van erre szükség, mert később ez a cím fog bekerülni a klienseknek válaszolt SDP
	üzenetekbe. 
\end{itemize}

Ezt a konfigurációs fájlt érdemes létrehozni úgynevezett ConfigMap segítségével, amit fellehet
használni a kontroller kapszulájának leírása során. Egy ilyen ConfigMap-ben kis fájlokat 
lehet létrehozni és azokat hozzáadni különböző telepítő definíciókhoz. 

A következő kódrészletben látható, hogy Kubernetes-n belül ezt a konténert hogyan kell 
létrehozni és elindítani benne a kontrollert. 

\begin{lstlisting}[caption=Kubernetes konténer specifikációja, label=lst:kubeSpec]
...
spec:
  volumes:
    - name: controller-volume
        configMap:
        name: controller-config
  containers:
    - name: rtpe-controller
      image: vidarhun/rtpe-controller
      volumeMounts:
        - name: controller-volume
          mountPath: /app/config
      command: ["python"]
      args: ["controller.py", "-c", "config/config.conf", "-l", "debug"]
...
\end{lstlisting}

A \textit{volumes} részben került betöltésre a \textit{controller-config} ConfigMap, ami 
tartalmaz egy a kontroller számára megfelelő konfigurációt. Majd a \textit{volumeMounts}-ban
kerül felcsatolásra a konténernek. Így már elérhető lesz a controller-config-ban meghatározott
fájl az \textit{/app/config} mappában a konténeren belül. Végül pedig a \textit{command} és az
\textit{args} paraméterekkel indul el maga a kontroller, ahol \textbf{-c} argumentummal 
lehet a konfigurációs fájl elérését megadni és az \textbf{-l} argumentummal lehet állítani,
hogy milyen szinten naplózzon a kontroller.

\section{Kontroller részei}

A kontroller fő részei közé tartozik az, hogy a Kubernetes API-val 
hogyan történik a kommunikáció, vezérlőprotokoll parancsainak megalkotása és
a beérkező üzenetek feldolgozása. 

\subsection{Kubernetes API}

Ahogy a Használt eszközök bemutatása fejezetben is lehet olvasni a Kubernetes
fürtnek parancsokat a Kube API-n keresztül lehet adni. Ezért kellett egy olyan
módszert találni, amivel a kontroller képes kommunikálni a Kubernetes API-val 
egy kapszulából. Ehhez használtam egy olyan könyvtárat, amivel
könnyen megvalósítható az API-val való kommunikáció. Mindemellett nagyon jól
van dokumentálva ez a könyvtár, így viszonylag egyszerűen használható.

A könyvtár telepítése után inicializálni kell a kapcsolatot a Kubernetes 
API-val, amihez a \textit{client} és \textit{config} modult kell importálni
a Kubernetes könyvtárból. Majd a \textit{config.load\_incluster\_config()}
hívással lehet inicializálni az API-val való kapcsolatot. Így jelezzük, hogy a
szkript egy fürtön belül egy kapszulában fog futni. De ahhoz ez ténylegesen jól 
működjön ahhoz konfigurálni kell az RBAC-t (Role-based access control), amivel 
olyan jogokat lehet biztosítani, amikkel hozzálehet férni Kubernetes 
erőforrásokhoz. Szabályozni lehet ezzel azt is, hogy egy adott erőforráshoz 
csak olvasási joggal rendelkezik a szkript így biztonságosabbá tehető a használata.

A kontrollerhez alább látható YAML definíció létrehozásával lehet olyan jogokat
adni, amikkel nagyjából bármilyen művelet eltud látni az l7mp által specifikált 
erőforrásokon.

\begin{lstlisting}[caption=RBAC létrehozása, label=lst:rbac]
apiVersion: rbac.authorization.k8s.io/v1
kind: ClusterRole
metadata:
  name: rtpe-controller
rules:
  - apiGroups: ["l7mp.io"]
    resources: ["virtualservices", "targets", "rules"]
    verbs: ["get", "list", "watch", "create", "update", "patch", "delete"]
---
apiVersion: rbac.authorization.k8s.io/v1
kind: ClusterRoleBinding
metadata:
  name: rtpe-controller
subjects:
  - kind: ServiceAccount
    name: default
    namespace: default
roleRef:
  kind: ClusterRole
  name: rtpe-controller
  apiGroup: rbac.authorization.k8s.io
\end{lstlisting}

A \textit{ClusterRole} segítségével lehet névtértől függetlenül jogokat biztosítani,
amik ebben az esetben az l7mp által specifikált egyéni erőforrásokra vonatkoznak. 
Ezeket az erőforrásokat az \textit{apiGroups} illetve a \textit{resources} listával
lehet megadni. Majd a \textit{verbs} listában meghatározott műveletekre kap 
jogok a kontroller. Így képes adott erőforrás állapotát lekérni, listázni, figyelni,
erőforrásokat létrehozni, frissíteni, kiegészíteni és törölni. 

Majd ezt követően a \textit{ClusterRoleBidning} erőforrással lehet egy adott
felhasználóhoz hozzárendelni a létrehozott jogokat. Ebben az esetben az alapértelmezett
felhasználó kapja meg ezeket a jogokat tekintve, hogy a Kubernetesben csak ez az egy
felhasználó van definiálva. Végül a \textit{roleRef}-ben lehet megadni, hogy mely
jogokat kapja meg a definiált felhasználó. 

Most, hogy már inicializálva van a Kubernetes API, létre kell hozni egy olyan 
API objektumot a kontrollerben, amivel egyéni erőforrás definíciókat lehet létrehozni
és törölni a fürtben. Ezt a következő sor használatával lehet létrehozni: 
\textit{api = client.CustomObjectsApi()}. \\

Így most már szabadon lehet az l7mp objektumait létrehozni és törölni a kontrollerből is, 
ami az előzőleg létrehozott \textit{api} objektum \textit{create\_namespaced\_custom\_object()}
illetve a \textit{delete\_namespaced\_custom\_object()} segítségével valósítható meg. Az elsőnek
említett függvény az alább felsorolt paramétereket várja.

\begin{itemize}
	\item \textbf{group}: A létrehozni kívánt erőforrás API címe.  
	\item \textbf{version}: A \textit{group}-ban definiált API verziója. 
	\item \textbf{namespace}: Melyik névtérben legyen létrehozva az erőforrás.
	\item \textbf{plural}: Az erőforrás típusának többesszáma. Ez az előzőleg definiált
	ClusterRole resources listájából valamelyik elem. 
	\item \textbf{body}: Egy YAML definíciót kell megadni python szótárként. 
\end{itemize}

Ezek közül a legfontosabb body, azaz a törzs, ahova mindig a híváshoz szükséges definíciókat
kell megadni. Ahhoz, hogy kódban ne legyenek ezek a definíciók annyira beégetve, ahhoz
ezeket a YAML fájlokat kiszerveztem egy mappába és, mikor használni vagy módosítani kell,
akkor csak a szükséges fájlt beolvassa a kontroller és módosítja a szükséges mezőket.

Ha egy hívás beérkezik a kontrollerhez, akkor kettő virtuális szolgáltatás és kettő
szabály fog létrejönni. A szolgáltatások mindig az l7mp-ingress-n fognak beállítódni, 
ugyanis ezek nyitják ki a rtpengine által meghatározott portokat a kliensek számára
a fürtön. A két szabály pedig az rtpengine kapszulák melletti l7mp proxy-ban 
fog létrejönni, ahol a már előre definiált szabálylistához adódnak hozzá. 

Az így létrehozott erőforrásoknak a neveit tárolni kell, mivel a hívás vége után ezeket
a létrehozott erőforrásokat törölni kell mindig, ahhoz pedig szükséges az erőforrás 
neve. 

\subsection{Konfigurálás és naplózás}

A kontroller indítása után az első feladata az a kódnak, hogy beolvassa az indítás
során megadott argumentumokat és feldolgozza az konfigurációs fájlban megadott 
paramétereket. Az argumentumok beolvasását a \textit{argparse} könyvár használatával
oldottam meg, mivel egy könnyel használható és bővíthető. Így a későbbiekben, ha 
több argumentum átadására van szükség nem lesz nagy feladat a megvalósítása. Másfelől
a konfigurációs fájl feldolgozásához is létezik egy könyvtár, amit \textit{configparser}-nek
hívnak és a Microsoft Windows INI fájljaihoz hasonló konfigurációkat képes beolvasni. Beolvasás
után feloldásra kerülnek az olyan paraméterek értékei, amik tartalmazhatnak tartomány neveket
is. Erre azért van szükség, mert létrehozott foglalatokkal csak konkrét IP címekre lehet 
csomagokat küldeni. A beolvasott paraméterek egy globális szótár objektumba kerülnek, ami 
miatt ezek a paraméterek könnyedén elérhetőek lesznek bármelyik függvényből. Ezzel a fejlesztés
nagy mértékben megkönnyebbül és átláthatóbb lesz a kód is. \\

A naplózás egy kulcsfontosságú része a kontrollernek, mivel fejlesztéskor vagy használata
során az stdout-ra kiírt események nagyban segítik a problémák keresését illetve azok 
megoldását. Naplózáshoz a \textit{logging} nevezetű könyvtárat használom, amiatt mert 
sok más könyvtár is ezt használja így, ha azok is valamit naplóznak, akkor az egyben 
látható lesz. Inicializálása során megadható, hogy milyen szinten naplózza az eseményeket 
és az is, hogy milyen formátumba. Ezért a kontroller egy naplóbejegyzése az alább látható 
módon néz ki. Ami a másodpercre pontos időből, a naplózás szintjéből illetve magából az 
üzenetből áll.

\begin{lstlisting}[caption=Naplózás formátuma, label=lst:logging]
15:41:55 [INFO] Started!
15:41:55 [INFO] Configuration file loaded!
\end{lstlisting}

\subsection{Foglalatok létrehozása}

Mivel a kontroller három protokollt képes jelenleg támogatni ezért háromféle foglalatot 
kell tudnia használni. Ami az UDP, TCP, WebSocket protokollokat támogatják. Ezeknek a
létrehozására mind van egy-egy függvény, ami képes a hozzájuk tartozó beállításokat 
beállítani és a megfelelő foglattal visszatérni, hogy azt több helyen lehessen használni. \\

A legegyszerűbb ebben az esetben az UDP foglalat megalkotása. Azért ez a legegyszerűbb,
mert ennél nem kell figyelembe venni a célját, mint a TCP esetében. Egy UDP foglalat 
létrehozáshoz szükség van a lokális címre és portra, amikhez a foglalatot hozzá kell 
kötni. A hozzákötés után, ha ezen a foglalaton keresztül történik csomagtovábbítás, 
akkor a megadott cím és port lesz a forrás cím és port. Emellett beállításra is
kerül egy időtúllépés paraméter is, ami 10 másodperc lesz. Ez azt jelenti, hogy
mikor a foglalat éppen csomagokat vár, akkor 10 másodpercig fog várni, majd generál 
egy kivételt és visszalép. Így használható egy olyan ciklusban, ami időközönként
valamilyen más feladat ellátását is megtudja valósítani. \\

Aztán van a TCP foglalat, aminél már figyelembe kell venni azt, hogy a WebSocket
protokollt használó kliens számára lesz gyártva vagy szimpla TCP kommunikációhoz 
használják. Ugyanis, ha önmagában kerül felhasználásra, akkor hozzáköti magát
az adott lokális címhez és porthoz és lényegében ugyan úgy kerül felhasználásra,
mint UDP esetben. Viszont, ha WebSocket kliens számára van létrehozva, akkor ezt
nem teszi meg, de ehelyett csatlakozik az rtpengine-hez az adott címen. Azt, hogy
miért van szükség a WebSocket számára egy TCP foglalatra azt a következő részben 
fejtem ki bővebben. De a csatlakozás az rtpengine-hez azért szükséges ebben az
esetben, mert másképp a kliens nem tudja TCP-nél jellemző kézfogás folyamatot
teljesíteni az rtpengine-l. \\

Végül pedig van a WebSocket verzió, amikor egy WebSocket foglalat kerül létrehozásra.
De ennek a létrehozására nem az általában használt python könyvtárat a \textit{socket}-t
fogja használni, hanem a \textit{websocket} könyvtárnak a \textit{create\_connection()}
függvényét. Ezzel a függvény egy olyan foglalattal tér vissza, aminél az elküldött 
üzenetek rendelkezni fognak a létrehozáskor beállított fejlécekkel. A következő 
függvény az, ami a kontrolleren belül létrehozza a WebSocket foglalatot. 

\begin{lstlisting}[language=python, caption=WebSocket foglalat létrehozása, label=lst:wssock]
def create_ws_socket(sock, header):
	return create_connection(
		f'ws://{config["rtpe_address"]}:{config["rtpe_port"]}',
		subprotocols=["ng.rtpengine.com"],
		origin=config['local_address'],
		socket=sock,
		header=header
	)
\end{lstlisting}

A következő listában sorrendben olvashatóak, hogy mely argumentum mit állít be és miért
pont azt.

\begin{itemize}
	\item Az rtpengine meghirdetett WebSocket URL-jét (Uniform Resource Locator) kapja 
	meg. Fontos, hogy ezt kapja, mivel az rtpengine egyszerre többféle protokollon keresztül tud 
	vezérlőparancsokat fogadni.
	\item Alprotokollokat is lehet használni WebSocket esetén, ami azért fontos, mert
	az rtpengine csak azokat a WebSocket csomagokat képes feldolgozni, amik rendelkeznek
	ezzel az \textit{ng.rtpengine.com} alprotokollal.
	\item A forrás címet lehet beállítani az \textit{origin} argumentummal. 
	\item Be lehet állítani azt, hogy a kapcsolat kiépítése során egyéni foglalatot
	használjon, így könnyen kontrollálható, hogy a foglalat milyen paraméterekkel 
	rendelkezzen. Ezért van szükség arra, hogy a TCP foglalat megalkotása során
	figyelembe vegyük azt, hogy WebSocket kliens számára hozzuk létre vagy csak úgy
	általánosan.
	\item Meg lehet adni egyéni fejléceket is, amikben bármi tárolható egy adott
	méretig. Ha létrejön egy hívás, akkor itt beállítódott hívás azonosítója alapján
	fogja a mellette lévő l7mp proxy eldönteni, hogy mely rtpengine kapszula kapja
	meg az üzeneteket.
\end{itemize}

\subsection{Erőforrások létrehozása és törlése}

Arról, hogy az erőforrások létrehozása illetve törlése mikor történik a feldolgozás
fejezetben lesz szó. Ebben arról lesz, hogy hogyan működnek.

Mikor elindul a kontroller, akkor már a kezdetekkor van egy \textit{kubernetes\_apis}
lista, ami tartalmazza az összes Kubernetes API kliens objektumot. Így mindig van egy
pontos képe a kontrollernek arról, hogy milyen paraméterekkel lettek erőforrások 
létrehozva a működése során. Szóval egy új hívás esetén kettő új API kliens objektum 
fog bekerülni ebbe a listába, egy hívó fél adataival és egy a hívott fél adataival 
és törlésnél ez a kettő kerül törlésre a hívásazonosító alapján. 

Egy erőforrás létrehozása során elsőnek ebben a listában kell ellenőrizni azt, hogy
ilyen hívásazonosítóval már létezik-e erőforrás, ha nem létezik 0csak akkor lehet létrehozni. 
Erre azért van szükség, mert a Kamailio rendelkezik egy válaszidővel, ami ha lejár
akkor az utoljára kiküldött üzenetet megismétli. De ebben az esetben könnyen előfordulhat,
hogy később küldi vissza a kontroller az üzenetet, mint ez az időkeret és ilyenkor 
a duplikáltan küldött üzeneteket elveti a kontroller.

Ezután egy \textit{query} üzenetet kell küldeni az rtpengine felé, ugyanis az 
\textit{offer} és \textit{answer} üzenetekre kapott válaszban nem biztos, hogy ugyan
az a port lesz lefoglalva. De a \textit{query} üzenettel ez kiküszöbölhető, mert az 
erre adott válaszban mindig a megfelelő portok fognak szerepelni. A portokra azért van
szükség, mert az API kliens objektumok által létrehozott erőforrásokban ezek használva
vannak. Alább látható, hogy egy API kliens objektum miként kerül létrehozásra.

\begin{lstlisting}[language=python, caption=Kubernetes API kliens objektum létrehozása, label=lst:kubeAPI]
kubernetes_apis.append(
	Client(
		call_id=call_id,
		tag=tag,
		local_ip=address,
		local_rtp_port=client_port,
		local_rtcp_port=client_port + 1,
		remote_rtp_port=remote_port,
		remote_rtcp_port=remote_port + 1,
		without_jsonsocket=config['without_jsonsocket'],
		ws=ws
	)
)
\end{lstlisting}

Az első argumentummal lehet meghatározni a hívás azonosítóját és a másodikkal a félhez
tartozó címkét, ami lehet \textit{from-tag} és \textit{to-tag} is. Majd ezután jön a
klienshez tartozó cím, RTP és RTCP port. Aztán az rtpengine által a híváshoz lefoglalt
RTP és RTCP port, végül pedig a \textit{without\_jsonsocket} és \textit{ws} argumentumokkal
lehet még erőforrásokra vonatkozó paramétereket megadni.

Ha ez a konstruktor sikeresen lefutott, akkor az azt jelenti, hogy a fürtben létrejött
minden szükséges erőforrás és adott minden a média forgalom kezeléséhez. \\

A törlés ezzel szemben nagyon egyszerűen úgy történik, hogy a hívásazonosító alapján
\textit{kubernetes\_apis} listából meghívjuk az objektum \textit{delete\_resources()}
függvényét, ami az adott híváshoz kapcsolódó összes erőforrást törölni fogja.

\subsection{Parancsok feldolgozása}

A parancsok feldolgozása kétféle módon történhet, lehet az átlagos foglalatokkal, ami 
a TCP és UDP csomagok kezelését jelenti illetve lehet a WebSocket protokoll segítéségével 
is. Az előzőleg említett feldolgozás lehetővé teszi, hogy mind TCP és UDP
foglalatok használatával működjön, mert ezek a foglalatok küldés és fogadás szempontjából
teljesen ugyanazokat a függvényeket valósítják meg. \\

Az említett eljáráshoz tartozó függvény két foglalatot vár el argumentumként, amik közül 
a \textit{sock} az a foglalat lesz amelyik a kliensektől várja a bejövő forgalmat és az 
\textit{rtpe\_sock} lesz, amelyen keresztül kommunikálni fog a kontroller az rtpnengine-el. 
Szóval a sock-n beérkező adat tárolásra kerül abban a formában, amelyben érkezik szóval 
egy egyszerű karakterláncként és a feldolgozott verziója is, amikor karakterlánc bencode 
által kódolt része feloldásra és szótárba írásra kerül. Ez a feldolgozott változat hasznos 
akkor, amikor az erőforrásokat kell létrehozni, míg a nyers beérkezett üzenet továbbküldésre 
kerül az rtpengine felé, hiszen azon nem kell változtatni semmilyen paramétert.

Ha az rtpengine válaszolt az adott parancsra és ez a parancs nem tartalmazott semmilyen 
SDP üzenetet, akkor egyszerűen visszaküldésre kerül a kliensnek. De ha a parancs egy offer
vagy answer üzenet volt, akkor az rtpengine által adott válaszban szereplő SDP üzenetben
módosítani kell a kapcsolat kiépítéséhez szükséges paraméterben szereplő IP címet. Ugyanis
ez a cím minden esetben 127.0.0.1 lesz, ami nem jó az olyan SIP klienseknek, mint a Linphone.
A szóban forgó címet kell lecserélni a dolgozó csomópont IP címére. De az rtpengine 
válasza után még vizsgálni kell azt is, hogy milyen parancs érkezett be a kontrollerhez. 
Ha egy \textit{delete} üzenet érkezett, akkor a benne szereplő hívásazonosítóhoz tartozó
Kubernetes erőforrásokat mindenképpen törölni kell, hogy véletlenül sem maradjon nyitva
olyan port a fürtön, ami nincs használatban. A másik ilyen fontos üzenet az \textit{answer}.
Azért kell az ilyen paranccsal rendelkező üzenetekre jobban figyelni, mint az \textit{offer}-re,
mert csak ennek a megékezése után lehet a megfelelő erőforrásokat létrehozni és értesíteni
a klienseket arról, hogy megkezdhetik médiaforgalom adását. Ha már az offer során 
létrehozásra kerülnek ezek az erőforrások, akkor könnyen előfordulhat, hogy egy üres híváshoz
lefoglalt portok kerülnek beállításra az l7mp erőforrások számára, ami azt eredményezi, hogy
a hívás egyik résztvevője nem tud semmilyen forgalmat küldeni az rtpengine felé.