\pagenumbering{roman}
\setcounter{page}{1}

\selecthungarian

%----------------------------------------------------------------------------
% Abstract in Hungarian
%----------------------------------------------------------------------------
\chapter*{Kivonat}\addcontentsline{toc}{chapter}{Kivonat}

A telekommunikációs alkalmazások többségében hagyományos szervereken vannak
használva. Mivel ezek az alkalmazások általában nagy mennyiségű klienst szolgálnak 
ki így ezeknek a szervereknek a skálázása, karbantartása rengeteg munkát igényel,
mert nem lehet olyan egyszerűen egy szervert elvenni vagy hozzáadni anélkül, hogy
ezt a felhasználók valamilyen módon ne vegyék észre. Emellett a rosszul skálázott 
alkalmazások könnyen plusz költséget jelenthetnek az üzemeltető cég számára, mivel 
felülskálázás esetén fölöslegesen vettek meg vagy foglaltak le adott mennyiségű 
szervert. Vagy alul skálázás esetén a felhasználók nem kapják meg az elvárt minőséget
így nagy eséllyel egy konkurens cégre váltanak. 

Erre egy jó megoldás lehetne, ha ezeket az alkalmazásokat sikerülne megvalósítani
felhős Kurbernetes környezetben. Így biztosítva azt, hogy a karbantartások és
skálázások a felhasználók számára teljesen transzparensen történjenek.

Eddig ez nem volt teljes mértékben megvalósítható, viszont az L7mp proxy és 
szolgáltatásháló segítségével dinamikusan lehet a felhasználók számára portokat
nyitni a Kubernetes fürtön és transzformálni a beérkezett csomagok protokollját 
más protokollra. Az így létrehozott VoIP (Voice over IP) rendszer, azért nem volt eddig megvalósítható, mert nem létezett egy olyan
proxy és szolgáltatásháló, ami képes lett volna UDP (User Datagram Protocol) 
csomagokat érdemben továbbítani. Maga az UDP azért egy lényeges protokoll, mert a 
VoIP rendszerekben RTP (Real-time Transport Protocol) és 
RTCP (RTP Control Protocol) protokollú csomagokkal történik a videó- és 
hangtovábbítás, amik UDP-t használnak.

A félév során az volt a feladatom, hogy készítsek egy olyan L7mp proxyt és 
szolgáltatáshálót használó architektúrát és alkalmazást, amely képes demonstrálni
a szükséges komponenseket ahhoz, hogy a jövőben lehessen felhőben VoIP 
alkalmazásokat használni. Ehhez fejlesztettem egy kontrollert, ami az rtpengine
RTP proxyt képes irányítani az rtpengine vezérlőprotokollján keresztül egy
Kubernetes fürtben. Ez a kontroller képes a beérkező vezérlőüzenetek alapján 
L7mp szolgáltatáshálóban létrehozni a megfelelő erőforrásokat illetve az üzenetekben
szereplő információt is képes módosítani. 

A teszteléshez fejlesztésre került egy kezdetleges VoIP kliens, ami képes az 
rtpengine vezérlőprotokollján kommunikálni a Kubernetes fürtben lévő kontrollerrel, 
majd az így kiépült kapcsolaton médiaforgalmat generálni. Ezzel a klienssel 
erőforrástól függően tetszőleges számú teszt hívást lehet létrehozni. 

Végül az így létrehozott rendszert teszteltem minél magasabb számú hívással és a 
VoIP specifikus mutatók alapján kiértékeltem a rendszer által nyújtott lehetőségeket. 

\vfill
\selectenglish


%----------------------------------------------------------------------------
% Abstract in English
%----------------------------------------------------------------------------
\chapter*{Abstract}\addcontentsline{toc}{chapter}{Abstract}

Most telecommunications applications are used on traditional servers. Because these 
applications usually serve a large number of clients
so scaling and maintaining these servers requires a lot of work,
because you can’t so easily take away or add a server without the users noticed in
any way. In addition, the poorly scaled
applications can easily be an added cost to the operating company because
in the case of over scaling, a given quantity was unnecessarily bought or reserved
server. Or scaling below will prevent users from getting the quality they expect
so there is a good chance they will switch to a competing company.

A good solution to this could be to implement these applications
in a cloud Kurbernetes environment. Thus ensuring that the maintenance and
scaling should be completely transparent to users.

So far, this has not been fully feasible, but with the L7mp proxy and
service mesh can be used to dynamically provide users with ports
open on the Kubernetes cluster and transform the protocol of received packets to
other protocol. The VoIP (Voice over IP) system thus created has not been feasible so far 
because there was no one
proxy and service mesh that could have been UDP (User Datagram Protocol)
forward substantial packets. UDP itself is an essential protocol because
In VoIP systems RTP (Real-time Transport Protocol) and
RTCP (RTP Control Protocol) packets are used for video and
audio transmission which are using UDP.

During the semester, my job was to create an L7mp proxy and
a service mesh architecture and application that can demonstrate
the necessary components to enable cloud VoIP in the future
applications. To do this, I developed a controller which can control rtpengine
through it's control protocol within the Kubernetes cluster.
This controller is capable of creating the appropriate L7mp resources based on the 
incoming control messages and modify them if it's needed.

For testing, a rudimentary VoIP client has been developed that is capable of
communicate with the controller in the Kubernetes cluster on the rtpengine control 
protocol,
then generate media traffic on the connection thus established. With this client
depending on the computing resources, any number of test calls can be created.

Finally, I tested the system created in this way with as many calls as possible and 
I evaluated the possibilities provided by the system based on VoIP-specific indicators. 

\vfill
\selectthesislanguage

\newcounter{romanPage}
\setcounter{romanPage}{\value{page}}
\stepcounter{romanPage}