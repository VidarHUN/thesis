%----------------------------------------------------------------------------   
\chapter{Összefoglalás}
%----------------------------------------------------------------------------

A szakdolgozat megvalósítása során rengeteg új ismerettel lettem gazdagabb, 
amiket tudok majd kamatoztatni későbbi munkáim során. Emellett sikerült 
egy komplex architektúrát megalkotni, amivel Kubernetes alatt lehet 
VoIP hívásokat kezelni. Emellett sokkal magabiztosabb tudásom lett Kubernetes és 
python terén is. 

Dr. Rétvári Gáborral terveink között szerepel ennek a projektnek a továbbvitele, 
ami során új funkciókkal kerülne kibővítésre és sokkal nagyobb teljesítményt 
lehetne elérni. 

Terveink között szerepel, hogy az L7mp-t tudjuk a kernel névtérben futtatni vagy 
a szolgáltatáshálóban lecserélni az L7mp-t proxyt Envoy-ra proxyra, ami sokkal nagyobb teljesítményre képes. Ilyen módon valószínűleg sokkal több hívást lehetne párhuzamosan kiszolgálni jelentősebb minőségbeli romlás nélkül. 

Szeretnénk a kontrollert oly módon megvalósítani, hogy a külvilág számára elérhető
legyen és úgy viselkedjen, mint egy SIP szerver. Lehetővé téve, hogy a SIP szerver és az 
rtpengine között sokkal kevesebb komponens lenne, amin keresztül kommunikálni tudnak 
egymással. Ez a megoldás kiküszöbölhetné azt a problémát, hogy adott terhelés mellett a 
Kamailio már nem tud kommunikálni az rtpengine-l.

Amit még lehet módosítani a rendszeren azaz rtpengine olyan szintű fejlesztése, ami 
lehetővé teszi számunkra azt, hogy Kubernetes kompatibilisen tudjon futni. Ez azt jelenti,
hogy több rtpengine kapszula esetén mindig csak egy kapszula proxyja lenne felkonfigurálva
a híváshoz szükséges szabályokkal. Viszont, ha használt rtpengine kapszula valami miatt 
megszűnik, akkor képes legyen egy olyan értesítést adni valamelyik komponensnek ami 
felkonfigurálja a második rtpengine kapszulát is.

Összesítve ez egy nagyon izgalmas téma, amit szívesen folytatnék és fejlesztenék 
a mesterképzés során. 