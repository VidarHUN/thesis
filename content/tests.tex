%----------------------------------------------------------------------------   
\chapter{Mérések}
%---------------------------------------------------------------------------- 

A VoIP hívás monitorozásra vonatkozó mutatók a hívásfolyam médiaútvonalának bármely 
csomópontján mérhetőek. Alapvetően számítással és összesítéssel történő elemzésre 
használják, és néha valós idejű teljesítménykövetésre és javításra is. \\

Ebben a fejezetben ismertetésre kerülnek a használt mutatók, tesztek menete és
azok eredményeinek összehasonlítása.

\section{Mutatók}

A következő mutatók kiszámolhatóak az RTCP csomagok alapján vagy maga az rtpengine
biztosítja számunkra a hívás végeztével.

\subsection{MOS és R-Faktor}

VoIP hívások minőségének mérésre két elterjedt mértékegység van az R-Faktor (osztályozási faktor)
és a MOS (Mean Opinion Score). MOS értéket gyakrabban lehet látni, viszont az R-Faktor szükséges
ahhoz, hogy kilehessen számolni a MOS-t. Az R-Faktor egy olyan mutató, ami a késleltetés, jitter és
csomagvesztés mutatókból kerül kiszámításra. Az értéke egy 50-90 közötti szám, ahol 
az 50 rossz minőséget, míg a 90 kiválót jelent. 

A MOS a hang, videó és audiovizuális minőséget kifejező mértékegység az 
ITU-T P.800.1 szerint. A hallgatás, a beszéd vagy a párbeszéd minőségére utal, 
függetlenül attól, hogy szubjektív vagy objektív modellekből származnak-e. Ez a
mértékegység egy egytől ötig terjedő skálán mozog, ahol a legalacsonyabb a legrosszabb
és legmagasabb a legjobb minőség.

\begin{itemize}
	\item Nagyon jó: 4,3 -5,
	\item Rossz: 3,1 -3,6,
	\item Nem ajánlott 2,6 - 3,1
	\item Nagyon rossz: 1 - 2,6.
\end{itemize}

A \ref{eqMos} képletben lehet látni, a MOS kiszámítását, ahol az \textit{R} paraméter 
az R-Faktor.

\begin{equation} \label{eqMos}
\text{MOS} = (((R-60) \cdot (100-R) \cdot 0,000007R) + 0,035R + 1)
\end{equation}

\subsection{Késleltetés}

Ez azaz idő, ami alatt a csomag egyik klienstől eljut a másikig ezredmásodpercben (ms). 

\subsection{Körülfordulási idő}

Az idő ami szükséges a csomagoknak, míg egyik klienstől eljut a másikig és vissza. SIP
hívások esetében ez azaz idő, amíg egy tranzakció tart, ami azt jelenti, hogy az elküldött
csomagra mennyi idő múlva érkezik egy nyugta. A szakdolgozat további részeben RTT (Round Trip Time)
fogja jelölni az angol neve után. 

Mérése ennek is ezredmásodpercben történik, mivel időről beszélünk, ezért minél magasabb ez a
szám annál rosszabb a hívás minősége. Az audiotartalom esetében a felső határ 150 ms, ami felett
már nagyon gyenge minőségű a hívás. Befolyásolhatja ezt a számot többek között, a kliensek 
közötti távolság, összeköttetés minősége és sávszélesség. A mérések során ezek mind nem fognak
változtatni az RTT idején, mert a szerverek "egymás mellett" vannak nagy sebességű kábelekkel
összekötve. Ami befolyásolni fogja, azaz a csomag útjában lévő proxyk mennyisége és azok feldolgozó
képessége. 

Fontos, hogy késleltetést nem lehet vele megbízhatóan mérni általános környezetben, mivel a hívó
felek között más útvonalak jelenhetnek meg, ami aszimmetrikus RTT eredményez. 

Viszont az rtpengine ezt a számot nem adja meg konkrétan, szóval egy egyszerű számítást
el kell rajta végezni, hogy a valós RTT-t megkapjuk.

\begin{equation}
\text{RTT} = \frac{\text{rtpegnine} \rightarrow \text{RTT}}{1000} + \text{rtpengine} \rightarrow \text{jitter} \cdot 2 + 10
\label{calcrtt}
\end{equation}

\subsection{Csomagvesztés}

Azon csomagok száma melyek nem jutottak el a céljukig. Szóval egyik klienstől a másikig. Ez különböző
indokok miatt történhet. A következő felsorolás nem teljes \cite{measure}

\begin{itemize}
	\item Nem elegendő sávszélesség. 
	\item Túlterhelés miatt nem képesek a felek feldolgozni a csomagokat.
	\item Tűzfalak, proxyk eldobják a csomagokat. 
\end{itemize}

\subsection{Jitter}

A kapott csomagok késleltetésének változása a folyam során. Méréséhez a csomagok küldési intervallumát
kell összehasonlítani a kapottak intervallumával. Szóval, ha két csomagot kiküldésre kerül 30 ms 
különbséggel és a kapott csomag 50 ms különbséggel érkezik, akkor a jitter 20 ms. Ennél a mutatónál is
az előnyös, ha minél kisebb, mivel már egy kicsi jitter és könnyen ronthat a hívás minőségén. 
Amik okozhatják:

\begin{itemize}
	\item A keret nagyobb, mint a jitter puffer mérete. 
	\item Algoritmusok késleltethetik az interfészeken beérkező csomagokat. 
	\item A kommunikáció közben lévő hálózati eszközök, proxyk feldolgozó képessége. 
\end{itemize}

Javítani úgy lehet ezen a számon, ha az alkalmazások használnak jitter puffert, ami elfogja a
beérkező csomagokat és azokat egyenletesen adja tovább feldolgozásra. 