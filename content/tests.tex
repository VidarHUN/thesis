%----------------------------------------------------------------------------   
\chapter{Mérések és kiértékelés}
%---------------------------------------------------------------------------- 

%Ebben a fejezetben ismertetésre kerülnek a használt mutatók, tesztek menete és
%azok eredményeinek összehasonlítása.

\section{Mutatók}

A következő mutatók kiszámolhatóak az RTCP csomagok alapján vagy maga az rtpengine
biztosítja számunkra a hívás végeztével.

\subsection{MOS és R-Faktor}

VoIP hívások minőségének mérésre két elterjedt mértékegység van az R-Faktor (osztályozási 
faktor) és a MOS (Mean Opinion Score). MOS értéket gyakrabban lehet látni, viszont az 
R-Faktor szükséges ahhoz, hogy ki lehessen számolni a MOS-t. Az R-Faktor egy olyan 
mutató, ami a késleltetés, Jitter és csomagvesztés mutatókból kerül kiszámításra. Az 
értéke egy 50-100 közötti szám, ahol az 50 rossz minőséget, míg a 90 és afeletti szám 
kiválót jelent. 

A MOS a hang, videó és audiovizuális minőséget kifejező mértékegység az 
\texttt{ITU-T P.800.1} szerint. A hallgatás, a beszéd vagy a párbeszéd minőségére utal, 
függetlenül attól, hogy szubjektív vagy objektív modellekből származnak-e. Ez a
mértékegység egy egytől ötig terjedő skálán mozog, ahol a legalacsonyabb a legrosszabb
és legmagasabb a legjobb minőség. A \ref{tab:mosValues} táblázatban olvasható a skálája. 

\begin{table}[ht]
	\footnotesize
	\centering
	\begin{tabular}{l c c c c c c}
		\toprule
		 & Nagyon jó & Jó & Átlagos & Átlagtól rosszabb & Nagyon rossz & Nem ajánlott\\
		\midrule
		MOS & 5 - 4,3 & 4,3 - 4,0 & 4,0 - 3,6 & 3,6 - 3,1 & 3,1 - 2,6 & 2,6 - 1,0\\
		\bottomrule
	\end{tabular}
	\caption{MOS szintjei}
	\label{tab:mosValues}
\end{table}

%\begin{itemize}
%	\item Nagyon jó: 4,3 -5,
%	\item Rossz: 3,1 -3,6,
%	\item Nem ajánlott 2,6 - 3,1
%	\item Nagyon rossz: 1 - 2,6.
%\end{itemize}

%A \ref{eqMos} képletben lehet látni, a MOS kiszámítását, ahol az \textit{R} paraméter 
%az R-Faktor.
%
%\begin{equation} \label{eqMos}
%\text{MOS} = (((R-60) \cdot (100-R) \cdot 0,000007R) + 0,035R + 1)
%\end{equation}

A MOS rendelkezik alkategóriákkal is, ami a hívások különböző részeinek minőségét méri. A 
mérések során ebből csak kettő lesz felhasználva, a MOS-CQ (MOS Conversational Quality) 
és MOS-LQ (Listening quality). Az előbbi a társalgás minőségét míg az utóbbi a hallgatás 
minőségét mutatja meg.

Azok a tényezők, amik befolyásolják a MOS és R-Faktor értéké a \ref{tab:values} 
táblázatban vannak összefoglalva. 

\begin{table}[H]
	\footnotesize
	\centering
	\begin{tabular}{ l c c c}
		\toprule
		Mutató & Jó & Átlagos & Rossz\\
		\midrule
		Jitter & $\leq$ 10 ms & 10 ms - 30 ms & $\geq$ 30 ms\\
		Csomagvesztés & < 0.5 \%  & 0.5 \% – 0.9 \% & $\geq$ 0.9 \%\\
		Hangszint & > -40 dB & -80 dB - -40 dB & < -80 dB\\
		RTT & < 200 ms & 200 ms - 300 ms & > 300 ms\\
		\bottomrule
	\end{tabular}
	\caption{MOS számításnál használt mutatók jó és rossz tartománya.}
	\label{tab:values}
\end{table}


\subsection{Késleltetés}

Azaz idő, ami alatt a csomag egyik klienstől eljut a másikig ezredmásodpercben (ms). 
Ha ennek a mutatónak az értéke 50 ms alatti, akkor a rendszerünk megfelelően működik. 
Addig, amíg ez a mutató 200 ms alatt van, addig jónak tekinthető a rendszer 
működése, viszont e felett már kellemetlen lehet a felhasználók számára a hívás. 

\subsection{Körülfordulási idő}

Az idő, ami szükséges a csomagoknak, míg egyik klienstől eljut a másikig és vissza. SIP
hívások esetében ez azaz idő, amíg egy tranzakció tart, ami azt jelenti, hogy az elküldött
csomagra mennyi idő múlva érkezik egy nyugta. A szakdolgozat további részeben RTT (Round 
Trip Time) fogja jelölni az angol neve után. 

Mérése ennek is ezredmásodpercben történik, mivel időről beszélünk, ezért minél magasabb 
ez a szám annál rosszabb a hívás minősége. Befolyásolhatja ezt a számot többek 
között, a kliensek közötti távolság, összeköttetés minősége és sávszélesség. A mérések 
során ezek mind nem fognak változtatni az RTT idején, mert a szerverek "egymás mellett" 
vannak nagy sebességű kábelekkel összekötve. Befolyásoló tényezők az útban lévő L7mp 
proxyk mennyisége és feldolgozó képessége lesz. 

Fontos, hogy késleltetést nem lehet vele megbízhatóan mérni általános környezetben, mivel 
a hívó felek között más útvonalak jelenhetnek meg, ami aszimmetrikus RTT eredményez. 

%Viszont az rtpengine ezt a számot nem adja meg konkrétan, szóval egy egyszerű számítást
%el kell rajta végezni, hogy a valós RTT-t megkapjuk.
%
%\begin{equation}
%\text{RTT} = \frac{\text{rtpegnine} \rightarrow \text{RTT}}{1000} + \text{rtpengine} \rightarrow \text{jitter} \cdot 2 + 10
%\label{calcrtt}
%\end{equation}

\subsection{Csomagvesztés}

Azon csomagok száma melyek nem jutottak el a céljukig. Ez különböző indokok miatt 
történhet. A következő felsorolás nem teljes \cite{measure}

\begin{itemize}
	\item Nem elegendő sávszélesség. 
	\item Túlterhelés miatt nem képesek a felek feldolgozni a csomagokat.
	\item Tűzfalak, proxyk eldobják a csomagokat. 
\end{itemize}

\subsection{Jitter}

A kapott csomagok késleltetésének változása a folyam során. Méréséhez a csomagok küldési 
intervallumát kell összehasonlítani a kapottak intervallumával. Például abban az esetben, 
ha két csomag 30 ms különbséggel kerül küldésre és a kapott csomag 50 ms különbséggel 
érkezik, akkor a Jitter 20 ms. Ennél a mutatónál is az előnyös, ha minél kisebb, mivel 
már egy kicsi Jitter is könnyen ronthat a hívás minőségén. Az alábbi felsorolásban 
olvasható mik okozhatják ennek a mutatónak a romlását:

\begin{itemize}
	\item A keret nagyobb, mint a Jitter puffer mérete. 
	\item Az interfészeken történő csomagfeldolgozás késleltetheti a beérkező csomagokat.
	\item A kommunikáció közben lévő hálózati eszközök, proxyk feldolgozó képessége. 
\end{itemize}

Javítani úgy lehet ezen a számon, ha az alkalmazások használnak Jitter puffert, ami 
elfogja a beérkező csomagokat és azokat egyenletesen adja tovább feldolgozásra. 

\section{Tesztek menete és értékelése}

A tesztek oly módon történtek, hogy az általam írt kliens generált egy adott számú 
háttérforgalmat, majd a Linphone kliensekkel elindítottam egy teljes értékű hívás, amivel 
könnyen lehetett a hívás minőségével kapcsolatos információkat kapni. Ahhoz, hogy ez a 
rendszer megfelelően tudjon működni a VirtualBox segítségével egy hasonló környezetet 
teremtettem, mint a hagyományos hívásfelépítés részben.

Az így létrehozott Kamailio és két Linphone kliens egy-egy virtuális gépben futottak és 
ezek a gépek bridged hálózatot használtak, amelyek a \ref{fig:servers} ábrán látható 
Tethys \texttt{enp12s0} és \texttt{enp17s0} interfészekhez van hozzákötve. 

A tesztelés során elsőnek próbáltam minél nagyobb háttérforgalom mellett egy normál 
hívást elindítani, viszont ez nem sikerült már 60-70 hívás esetén. Tehát ez lett a 
maximum kezelhető hívások száma. A probléma nem feltétlen a Kubernetes fürt 
túlterhelésében történt, hanem az L7mp proxyk kapacitásának elérése miatt. Mikor egy 
teljes értékű Kamailio + Linphone által kezdeményezett hívást próbáltam indítani már nem 
volt képes a Kamailio ellenőrizni az rtpengine elérhetőségét egy \texttt{ping} üzenettel, 
ezért a klienseknek egymás elérhetőségét adta és nem szúrta be közéjük az rtpengine-t.

A diagramokon, táblázatokban szereplő adatok hálózati mérésekből lettek kinyerve. Ezeket 
a méréseket minden alkalommal az egyik Linphone kliensnél gyűjtöttem a \texttt{tcpdump} 
nevezetű program segítségével. Ezzel a programmal csomagokat lehet elkapni és analizálni. 
Az így kigyűjtött csomagokat végül \texttt{WireShark} segítségével analizáltam. A 
\texttt{WireShark} lényegében ugyanazokat a funkciókat nyújtja, mint a \texttt{tcpdump}, 
azzal a különbséggel, hogy a \texttt{WireShark} rendelkezik grafikus felülettel, ami 
könnyebben használható. 

\subsection{Mérések}

A legreprezentatívabb mutató a MOS, aminek a társalgási és hallgatósági minőségét lehet 
látni a \ref{fig:moslq} ábrán.

\begin{figure}[!ht]
	\centering
	\includegraphics[width=1\textwidth, keepaspectratio]{figures/mos.png}
	\caption{MOS-LQ változása az hívás során adott változó háttérforgalmak mellett.}
	\label{fig:moslq}
\end{figure}

Csak a \texttt{MOS-LQ} alakulása került ábrázolásra, mert a \texttt{MOS-CQ}-val szemben 
csak nagyon kevés különbség van. Látható, hogy adott mennyiségű háttérhívás esetén mennyire romlik a teljes értékű Linphone hívásnak a minősége.

%\begin{figure}[!ht]
%	\centering
%	\includegraphics[width=1\textwidth, keepaspectratio]{figures/moscq.png}
%	\caption{MOS-LQ változása az hívás során adott változó háttérforgalmak mellett.}
%	\label{fig:moscq}
%\end{figure}

A \ref{fig:moslq} ábrán látható, hogy a MOS minimális értéke az jóval kisebb minden 
esetben, mint az átlaga. Ez annak köszönhető, hogy a hívás kezdetekor az L7mp proxyk nem 
képesek elég gyorsan feldolgozni a csomagokat így a magas Jitter keletkezik, ami rossz 
irányba befolyásolja a MOS értéket. Ezzel szemben a sokkal jobb átlagos és maximális MOS 
érték annak köszönhető, hogy a Linphone kliens alkalmaz egy adaptív Jitter puffert, 
aminek a méretét annak megfelelően állítja, hogy mekkora Jitter keletkezik a hívás során. 

%Ennek a Jitter puffernek az alakulása látható a 
%A \ref{fig:moslq} ábrán jól látható az folyamat, hogy a hívás kezdte után mennyire rossz 
%a hívás minősége illetve, hogy egy idő után mennyire javul és vesz fel egy viszonylag kis 
%kilengésekkel rendelkező értéket. A hívás elején tapasztalható rossz minőséget az okozza, 
%hogy az okozza, hogy az L7mp proxyk nem képesek elég gyorsan feldolgozni a csomagokat így 
%a magas Jitter keletkezik a rendszerben, amikre nincs felkészülve egyik Linphone kliens 
%sem. Később azért lesz jobb a MOS értéke, mert a kliensek adaptív Jitter pufferel 
%rendelkezni így idővel megtanulják azt, hogy mekkora az átlagos Jitter és annak 
%megfelelően növelik a pufferük méretét, ezzel kisimítva a csomagok érkezését. 

De ez nem jelenti azt, hogy a hívás maradéktalanul jó lesz, amit mutat a
\ref{fig:voiceComp} ábra is, amit a \texttt{WireShark} RTP folyam elemzőjéből mentettem 
ki. 

\begin{figure}[!ht]
	\centering
	\includegraphics[width=0.7\textwidth, keepaspectratio]{figures/calls60.png}
	\includegraphics[width=0.7\textwidth, keepaspectratio]{figures/calls20.png}
	\caption{MOS-LQ változása az hívás során adott változó háttérforgalmak mellett.}
	\label{fig:voiceComp}
\end{figure}

A \ref{fig:voiceComp} ábrán a felső hullám 60 hívás mellett készült, míg az alsó 20 hívás 
mellett keletkezett. Ezek azok a RTP folyamok, amik az rtpengine felől érkeztek a 
kliensek felé. 

Jól látszik, hogy a felsőnél rengeteg csend van beszúrva, ami nagyon recsegőssé teszi a 
hívást és élvezhetetlenné, míg az alsónál csak az elején található egyszer egy magas 
Jitter érték, ami nem befolyásolja annyira a hívás minőségét.

\begin{table}[ht]
	\footnotesize
	\centering
	\begin{tabular}{l c c c c c c c}
		\toprule
		 & 0 & 10 & 20 & 30 & 40 & 50 & 60\\
		\midrule
		MOS-LQ & 4,7 & 4,4 & 4,5 & 4,2 & 3,9 & 2,8 & 3,1\\
		MOS-CQ & 4,7 & 4,4 & 4,5 & 4,2 & 3,9 & 2,8 & 2,9\\
		R-Faktor & 94 & 88 & 90 & 84 & 78 & 55 & 59\\
		RTT & 17 ms & 9 ms & 17 ms & 16 ms & 16 ms & 42 ms & 182 ms\\
		Jelszint & -3 & -2 & -4 & -3 & -2 & -8 & -2\\
		Zajszint & -11 & -11 & -12 & -13 & -10 & -22 & -10\\
		Jitter & 2 ms & 7 ms & 13 ms & 5 ms & 66 ms & 69 ms & 173 ms\\
		Max Jitter puffer & 104 ms & 41 ms & 75 ms & 64 ms & 75 ms & 89 ms & 234 ms\\
		\bottomrule
	\end{tabular}
	\caption{Adott háttérhívás mellett mért mutatók átlagai.}
	\label{tab:callValues}
\end{table}

A \ref{tab:callValues} táblázatban jól látszik, hogy a hívások minősége miként romlott 
annak függvényében, hogy hány háttérhívás volt egyszerre. Rögtön észrevehető, hogy a 
MOS értéke az folyamatosan rosszabb attól függetlenül, hogy a \ref{fig:moslq} ábrán az 
látható, hogy idővel javul ennek az értéke. Ez attól van, hogy itt már az összes  
értelmezhető RTCP üzenet fel lett dolgozva és egy idő után beállt a MOS értéke egy adott 
szintre, ezért folyamatos csökkenés látható. 

A másik ilyen könnyen jellemezhető érték azaz R-Faktor, ahol jól látszik, hogy nulla 
háttérhívás mellett átlagosan sikerült elérni azt a minőséget, ami majdnem tökéletesnek 
számít. Ezt a viszonylag jó értéket tartja egészen 40 hívásig, ahol is már egy nagy esés 
látható. Ez már azt jelenti, hogy élvezhetetlen a hívás. Ennek kiváltó oka lehet a magas 
Jitter, ami már ennyivel képes torzítani a hívást. A Jitter egész tolerálható értékeket 
adott 40 hívásig, viszont azután már túl nagyok voltak, mivel 60 ms alatt nagyjából három 
RTP csomagnak kellett volna érkeznie.

A Linphone 60 ms-s Jitter pufferrel rendelkezik alap esetben és, ha szükség van rá, akkor 
ezt képes módosítani. Ez látszik a táblázat utolsó sorában is, mivel adaptívan növeli 
puffer méretet.

\section{Rugalmasság mérése}

A rugalmasság mérése során az fog kiderülni, hogy egy rtpeninge meghibásodása, milyen 
módon befolyásolja a hívást. Ennek szimulálása érdekében a hívás indítása után 
megvizsgálom a Kubernetes fürtben szereplő rtpengine kapszulákat, hogy melyik kezdte 
el feldolgozni a hívást és azt a kapszulát törlöm. Az elvárt működés során annak kell 
történnie, hogy az L7mp operátor képes detektálni a kapszula törlését melynek során 
értesíti az L7mp proxykat, hogy az az rtpengine kapszula már nem elérhető. Ennek hatására 
az L7mp proxykban definiált terheléselosztó egy új végponthoz fogja továbbítani a 
beérkező RTP és RTCP csomagokat. 

A leírt eljárás során nem képesek az RTP és RTCP csomagok eljutni egyik rtpengine 
kapszulához sem, ezért kiesést kell tapasztalni. De ennek a kiesésnek minél rövidebb 
időnek kell lennie, különben a hívásban résztvevő felek nem hallanak bizonyos részeket. 
Ezenfelül okozhat még kapcsolatvesztést is a klienseknél, ha egy adott ideig nem érkezik 
semmilyen médiaforgalom. 

\begin{figure}[!ht]
	\centering
	\includegraphics[width=1\textwidth, keepaspectratio]{figures/res.png}
	\caption{Elvesztett beszédrészlet meghibásodás esetén}
	\label{fig:res}
\end{figure}

A \ref{fig:res} ábrán látható, hogy médiafolyam során hol keletkezett kiesés az rtpengine kapszula törlése után. Ez a kiesés összesen 1,1 másodperc kiesést okozott ami a beérkezett csomagok 1,4\% százalékának az elvesztését jelenti.

Összességében elmondható, hogy ezt a feladatot jól ellátja az így létrehozott architektúra, mivel egy ekkora kiesés még nem biztos, hogy kellemetlen lehet a felhasználók számára. Emellett a hívás minősége is megmaradt a törlés után. 