%----------------------------------------------------------------------------   
\chapter{Használt eszközök bemutatása}
%---------------------------------------------------------------------------- 

\section{L7mp}

Az L7mp egy több protokollt támogató szolgálatatás haló es porxy, amellyel 
könnyen lehet a legtöbb szállítás és alkalmazás réteg béli protokollt továbbítani
illetve fordítani. Esetünkben a szolgáltatás háló funkcionalitása lesz a 
fontosabb, de ettől függetlenül kitérek a proxy részére is, mert minden a 
legtöbb Kubernetes erőforrás használni fogja a proxy részét is. 

A szakdolgozat szempontjából még érdemes megemlíteni, hogy ez egy még nagyon 
erősen fejlesztés alatt álló projekt, szóval a vele megvalósított megoldások 
nem biztos, hogy tudják garantálni a megbízhatóságot. 

\subsection{L7mp, mint proxy}

Működésében, nagyon hasonló az Envoy-hoz, ami egy a lyft által készített és 
nagyon sok helyen használt proxy. Azzal a különbséggel, hogy az L7mp inkább 
az alsóbb rétegbeli protokollokat támogatja, ezért a szállítási réteg legtöbb 
protokollja támogatva van és lehet bizonyos értékeik alapján irányítani a 
proxy-n áthaladó forgalmat. Példának okáért lehet olyat csinálni, hogy egy
Unix Domain Socket (UDS) foglalaton figyeli a forgalmat, majd azt WebSocket-re
átfordítva küldi tovább. 

Ezzel szemben nagyon sok olyan tulajdonsággal nem rendelkezik, ami megtalálható
az Envoy-ban. Ilyennek tudható be, hogy az alkalmazás réteg protokolljai nincsenek 
teljesen támogatva vagy nem rendelkeznek olyan szintű implementációval. Ami 
még egy nagyobb hátránya, hogy nem tud olyan gyors lenni az L7mp, mint az Envoy.
Ennek a legnagyobb oka, hogy az L7mp Node.js keretrendszerben íródott, ami a 
JavaScript miatt sokkal lassabb, mint a C++ nyelven íródott Envoy.

\subsubsection{Felépítése}

Mint már említve volt az L7mp rendkívül hasonlít az Envoy-ra és ez nincs másképp 
a felépítésben is. Szóval ennek a proxy-nak a használatához a következő 
építőköveket kell jobban ismerni: \textbf{Session}, \textbf{Listeners}, \textbf{Clusters}, \textbf{Endpoints}, \textbf{Rules}, \textbf{Routes}.\\

A Session (munkamenet) mindig akkor jön létre, amikor valamelyik Listener 
bejövő forgalmat kap. Erre azért van szükség, hogy a láncban meghatározott 
elemeknek mindig a megfelelő függvénye hívódjon meg és a hibák megfelelően 
legyenek kezelve. 

A Listener vagy magyarul figyelő úgy működik, mint egy szerver, ami bizonyos 
szempontok alapján figyelik a hozzájuk beérkező forgalmat és azt továbbítják
egy adott címre. Ezek a címek a Cluster-ek (fürtök) szoktak lenni, amiknek a 
működése közel azonos a Listener-ével, mivel ezek is a rajtuk beérkező forgalmat
adják tovább, azzal a kivétellel, hogy ezek nem láthatóak a proxy-n kívülről.

Végük a forgalom utolsó állomása az Endpoint vagyis végpont, ami már egy konkrét
cím, ahol figyel az általunk beállított alkalmazás. Fontos megjegyezni, hogy 
ezek az útvonalak kétirányúak, szóval a proxy képes kimenő forgalmat is ugyanúgy 
kezelni, mint a bejövőt. 

Amik a fentebb említett elemeket összeköti az a Rule (szabály) és a Route (irány),
A szabály által lehet különböző paraméterekre szűrni vagy hozzáadni. Ez azért egy 
nagyon hasznos funkció, mert ezáltal egy Session-n belül képes UDP forgalmat 
címkékkel ellátni és ezen címkék alapján megfelelő irányba továbbítani a 
forgalmat.

\subsubsection{Programozása}

Az L7mp indításához mindig biztosítani kell neki egy alapkonfigurációt, ami 
lehet általában egy L7mpController Listener-t és egy Cluster-t jelent, amin 
keresztül információkat tudunk szerezni az éppen futó proxy-ról vagy új 
elemeket lehet egyszerűen létrehozni. 

Új elemek hozzáadásához csak a proxy címére kell egy POST HTTP (HyperText Transfer 
Protocol)hívást indítani, aminek keretében a konfigurációt YAML (YAML Ain't Markup 
Language) vagy JSON (JavaScript Object Notation) formátumban tudjuk átadni.
Erre alább lehet látni egy nagyon egyszerű példát, ahol az L7mp a következő címen 
figyel: http://localhost:1234. 

\begin{lstlisting}
	curl -iX POST --header 'Content-Type:text/x-yaml' --data-binary @- <<EOF  http://localhost:1234/api/v1/listeners
	listener:
	  spec:
	    protocol: WebSocket
	    port: 2000
	  rules:
	    - action:
	        route:
	          destination:
	            spec:
  	              protocol: UDP
	              port: 3000
	            endpoints:
	              - spec:
	                  address: 127.0.0.1
	EOF
\end{lstlisting}

A fenti hívásban az látható, hogyan lehet egy Listener-t minden attribútumával 
létrehozni a proxy-n belül. Ez most a 127.0.0.1:2000-s címen figyel és minden 
bejövő forgalmat UDP-n továbbít a 127.0.0.1:3000-s címre. De ettől sokkal több
mindent is belehet állítani könnyedén. 

\subsection{L7mp, mint szolgáltatás háló}